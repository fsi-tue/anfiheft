Eduroam ermöglicht es euch, weltweit an verschiedenen teilnehmenden
Universitäten\footnote{Deutschland-Karte:
\url{https://map.eduroam.de/leaflet/eduroam/eduroam-map.html}} die Zugangsdaten	%TODO insert \link{}{}?
deiner eigenen Uni zu benutzen, um ins Internet zu kommen. Dass eine Uni oder
FH bei Eduroam mitmacht, bekommt ihr i.\,d.\,R. relativ schnell mit, sobald ihr
euer Smartphone in die Hand nehmt.

\subsection*{Zertifikat}
  Um zu verhindern, dass jemand irgendwo ein WLAN-Netz mit dem Namen eduroam
  eröffnet und eure Daten abgreift, müsst ihr unbedingt das Zertifikat
  überprüfen\footnote{Genau genommen prüfen wir den Zertifizierungspfad mit dem
  Wurzelzertifikat der Deutschen Telekom.}. Dieser Schritt ist leider etwas
  umständlich, sollte aber auf keinen Fall weggelassen werden! Ihr braucht dazu das Wurzelzertifikat \glqq\texttt{T-Telesec Global Root Class 2}\grqq. Dies
  sollte schon vorhanden sein (v.\,a. unter GNU/Linux und Windows), kann
  jedoch unter Android nicht für W-LAN verwendet werden. Außerdem ist unter iOS
  und OS X laut ZDV eine manuelle Konfiguration nicht möglich, es muss also das
  Konfigurationsprofil vom ZDV verwendet werden
  \footnote{\url{https://uni-tuebingen.de/fileadmin/Uni_Tuebingen/Einrichtungen/ZDV/Dokumente/Anleitungen/eduroam/eduroam_2021.mobileconfig}},
  welches das Zertifikat bereits beinhaltet.	%TODO insert \link{}{}?

\subsection*{Konfiguration}
  Am besten folgt man einer der ausführlichen und plattformspezifischen
  Anleitungen vom ZDV:\\
  \link{https://uni-tuebingen.de/de/13958}{Zu den ZDV Anleitungen}\\
  Als fortgeschrittene Variante kann man unter GNU/Linux \texttt{netctl} oder auch
  direkt \texttt{wpa\_supplicant} verwenden. Anleitungen dazu findet man im
  \texttt{configs}-Repo der Fachschaft
  (\url{https://github.com/fsi-tue/configs/tree/master/eduroam}).	%TODO insert \link{}{}?

\subsection*{Authentifizierung}
\begin{minipage}{\textwidth}
  Um zu gewährleisten, dass der Zugang überall funktioniert, muss man sich an
  folgende Konvention halten (bitte beachtet, dass hier kein
  \emph{@student.uni-tuebingen.de} verwendet werden darf): \medskip \\
  \begin{tabular}{l|l}
    Benutzername (Identität) & <zx-Kürzel>@uni-tuebingen.de \\ \hline
    Passwort                 & <ZDV-Passwort>
  \end{tabular}
\end{minipage}

\vfill
