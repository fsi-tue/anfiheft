
Wenn ihr diesen Abschnitt interessiert lest, gehört ihr zu einer der aufstrebenden Nachwuchs-Generationen von Medizininformatikern. Weil Tübingen sowohl eine renommierte Medizin als auch Informatik hat, wurde zum vorletzten Wintersemester dieser Studiengang geboren. Im Gegensatz zum Wintersemester 2012/13 wurden anfängliche Schwierigkeiten behoben und der Einstieg ins Studium der Medizininformatik sollte nun, wie bei allen anderen Informatik-Studiengängen in Tübingen, mehr oder weniger problemlos verlaufen.
Für nähere Informationen zu den Medizininformatik-spezifischen Veranstaltungen
wendet euch bitte an die entsprechenden Personen aus der Fachschaftsliste auf
der letzen Seite dieses Anfihefts. :)

Als Medizininformatiker hört ihr nicht nur Vorlesungen aus der Informatik (welche einen Großteil der ersten beiden Semester ausmachen), sondern auch aus der Medizin, Mathematik, Physik und Biologie. Genauer gesagt:
\begin{itemize}
	\item Wie jeder am WSI durchlauft ihr auch das Standardprogramm in der Informatik, bestehend aus \textbf{Informatik I und II, Algorithmen} und dem \textbf{Teamprojekt}. Im Gegensatz zu anderen Informatikern lasst ihr die Informatik III aus. Da ihr später vielleicht an hoch technischen medizinischen Geräten rumschrauben müsst, hört ihr \textbf{Informatik der Systeme}. Ferner hört ihr \textbf{Einführung in die Internettechnologien}, \textbf{User Interface Design} um euch eine Vorstellung darüber zu machen, wie Programme im medizinischen Bereich bedienbar sein sollten. Veranstaltungen, bei denen ihr unter euch seid, sind \textbf{Grundlagen der Medizininformatik} sowie in den späteren Semester \textbf{Ökonomie in der Medizininformatik}, \textbf{Telemedizin} und \textbf{Medizinische Visualisierung}. Später dürft ihr dann noch 12 LP mit Wahlpflichtmodulen der Informatik füllen.
	\item Als Medizininformatiker habt ihr als Einzigen die Möglichkeit, vom oben genannten "`Standardprogramm"' abzuweichen und anstatt \textbf{Algorithmen} die Veranstaltung \textbf{Grundlagen der Bioinformatik} zu hören.
	\item Ihr fangt wie alle anderen mit \textbf{Mathematik I und II} an, lasst jedoch \textbf{Mathematik III-IV} weg.  Bei den Physikern hört ihr zudem noch \textbf{Experimentalphysik I+II}.
	\item Um bei den Medizinern und Biologen mitreden zu können, besucht ihr zusammen mit den Medizintechnikern aus Stuttgart die Vorlesungen \textbf{Humanbiologie I bis III}, des weiteren \textbf{Biostatistik}. Damit ihr Mediziner versteht (und euch in der Biologie einige Dinge herleiten könnt) auch noch \textbf{Medizinische Terminologie}. In der Medizin dürft ihr euch später auch bei Wahlpflichtmodulen austoben, insgesamt 6 LP gilt es hier zu holen.
	\item Weil ihr als halbe Mediziner sozial sehr kompetent sein müsst und die Möglichkeit haben sollt, über den Tellerrand hinauszuschauen, dürft ihr ganze 12 Punkte bei den \textbf{Überfachlichen berufsfeldorientierten Kompetenzen} holen.
	\item Noch ein \textbf{Hinweis in eigener Sache:} Wir bisherigen Medizininformatiker empfehlen euch an dieser Stelle, möglichst früh einen Studiengangssprecher zu wählen und euch mit den Medizintechnikern auszutauschen. Bisher kam es leider manchmal vor, dass die Medizinische Fakultät nichts von Medizininformatikern wusste, dies wird sich zukünftig wohl aber auch bessern.
\end{itemize}

Als Orientierungsprüfung müsst ihr die Veranstaltungen "`Informatik I"' oder "`Informatik II"' sowie "Humanbiologie I"' und "`Medizinische Terminologie"' bestehen.

% WS 2015/16 auf neue POs umgeschrieben - tim
