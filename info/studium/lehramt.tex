% studium/lehramt.tex

Informatik muss in einer 2-Fächer-Kombination (also als eines von 2 Hauptfächern) studiert werden. Auch eine Drei-Fächer-Kombination ist möglich, wobei das dritte Fach dann optional ist; dies kann auch ein Nebenfach sein (Unterricht bis 10. Klasse möglich), ist aber im Fall der Informatik nicht möglich, da Nebenfächer nur bis zur 10. Klasse unterrichtet werden dürfen. Näheres dazu ist auf der Seite des Ministeriums für Wissenschaft und Forschung, Baden-Württemberg (\url{http://www.studieninfo-bw.de}) zu finden, Stichwort "`Studienfächer"'.

Dort gibt es unter anderem eine Auflistung der möglichen Fächer\-kom\-bi\-na\-tio\-nen und die gymnasiale Prüfungsordnung (GymPO 2010), in der nachzulesen ist, was man für die Zulassung zum ersten Staatsexamen benötigt.

In der Zwei-Fächer-Kombination entfallen etwa 1/3 der Punkte auf Informatik, 1/3 auf das zweite Hauptfach, und 1/3 auf fächerübergreifende Inhalte wie Pädagogik, ethisch-philosophische Grundlagen, Module der personalen Kompetenz und die wissenschaftliche Abschlussarbeit (Zulassungsarbeit).
Für die Erfüllung der pädagogischen Studien müssen 2 Vorlesungen und 2 Seminare belegt werden (Veranstalter ist das Institut für Erziehungswissenschaft / Fakultät 08). Das ethisch-philosophische Grundlagenstudium umfasst zwei Seminare (EPG 1 und EPG 2), welche von den jeweiligen Fakultäten angeboten und vom IZEW\footnote{Interfakultäres Zentrum für Ethik in den Wissenschaften \url{http://www.izew.uni-tuebingen.de/epg/tue.html}} koordiniert  werden.
Die Module für personale Kompetenz werden vom ZfL (Zentrum für Lehrerbildung) angeboten.
Fächerübergreifend müssen außerdem ein Orientierungspraktikum und ein Praxissemester absolviert werden. Informationen dazu finden sich, zusammen mit FAQ und weiteren Unterlagen, auf der Universitäts-Homepage zum Informatik-Lehramt.
