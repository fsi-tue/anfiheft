Sollte man nicht in die Kategorie der BaföG-Empfänger fallen, kann man sich immer noch auf ein Stipendium bewerben. Stipendien werden meistens von Stiftungen ausgeschrieben, welche eine gewisse politische, gesellschaftliche oder religiöse Ausrichtung haben. Solltet ihr euch also überlegen, euch auf ein Stipendium zu bewerben, empfiehlt es sich, vorher zu überlegen, welche Stiftung zu euch passen würde. Hierzu kann man sich die Webseiten der verschiedenen Stiftungen anschauen oder Vergleichsseiten wie die des Deutsche Akademische Austauschdienst\footnote{\url{https://www.daad.de/de/studieren-und-forschen-in-deutschland/stipendien-finden/}} zu Rate ziehen. \medskip \\	%TODO insert \link{}{}?
Um gute Chancen auf ein Stipendium zu haben, solltet ihr eine sehr gute Note im Abitur haben und ehrenamtlich tätig sein.