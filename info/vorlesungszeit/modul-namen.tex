\subsection*{Modulnamen unter der Prüfungsordnung 2021}
Studierende die ab dem Wintersemester 2021 mit ihrem Studium begonnen haben,
werden häufiger über verschiedene Bezeichnungen stolpern.\\
Für die neue Prüfungsordnung wurden einige altbekannte Vorlesungen offiziell
umbenannt. Das hält Dozenten und ältere Studenten aber nicht davon ab, die
Veranstaltungen mit ihrem alten Namen zu bezeichnen.

\begin{tabular}{ll}
\emph{Alte Bezeichnung}						& \emph{Neue Bezeichnung} \\[.7em]
Mathematik 1											& Mathematik für Informatik 1: \\
																	& Analysis \\[.7em]
Informatik 1										  & Praktische Informatik 1: \\
																	& Deklarative Programmierung \\[.7em]
Einführung Technische Informatik  & Technische Informatik 1: \\
																	& Digitaltechnik \\[.7em]
Mathematik 2											& Mathematik für Informatik 2: \\
																	& Lineare Algebra \\[.7em]
Informatik 2											& Praktische Informatik 2: \\
																	& Imperative und objekt-orientierte Programmierung \\[.7em]
Informatik der Systeme						& Technische Informatik 2: \\
																	& Informatik der Systeme \\[.7em]
Basispraktikum										& Technische Informatik 3: \\
																	&	Praktikum Mikrocomputer \\[.7em]
Algorithmen												& Theoretische Informatik 1: \\
																	& Algorithmen und Datenstrukturen \\[.7em]
Mathematik 3											& Mathematik für Informatik 3: \\
																	& Fortgeschrittene Themen \\[.7em]
Software Engineering							& Praktische Informatik 3: \\
\footnotesize{War kein eigenes Modul}		& Software Engineering \\[.7em]
Theoretische Informatik						& Theoretische Informatik 2: \\
																	& Formale Sprachen, Berechenbarkeit und Komplexität \\[.7em]
Stochastik / Numerik							& Mathematik für Informatik 4: \\
																	& Stochastik oder Numerik \\[.7em]
Teamprojekt												& Praktische Informatik 4: \\
\footnotesize{Früher an Software Engineering geknüpft} & Teamprojekt
\end{tabular}

