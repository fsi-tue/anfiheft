Liebe Erstsemester,

was ihr zu einer ersten Orientierung im Studi-Alltag braucht, haben wir
versucht, auf den folgenden Seiten für euch zusammenzutragen.  Dieses Heft
ist daher vor allem eine Sammlung allerlei wichtiger Infos und Tipps\footnote{
Bitte beachtet: Auch wir können uns irren, rechtlich verbindlich ist nur die
jeweilige Studienordnung!}, die ihr
euch sonst selbst mühsam durch viele Fragen und Rennerei %von schwarzen Brettern, die quer über ganz Tübingen
%verteilt sind, oder aus Sekretariaten, deren erstaunliche
%Öffnungszeiten\footnote{Inzwischen haben wir für euch unter folgender Adresse 
%eine inoffizielle Übersicht zusammengestellt:\\
%\url{https://www.fsi.uni-tuebingen.de/studium/sekretariate}}
%auch für den Rest eures Studiums die ganz große Unbekannte bleiben werden,
zusammensuchen müsstet.

Unsere persönlichen Erfahrungen am WSI -- Wilhelm-Schickard-Institut -- sowie diverse Kontaktmöglichkeiten
haben wir natürlich auch eingebracht.  %Diese betreffen vor allem die
%Nebenfächer und damit das spannendste Thema eures Studienbeginns.

Ihr haltet bereits die \number\auflage. Auflage unseres Info-Heftes in den
Händen. Trotzdem mag das ein oder andere noch zu verbessern sein.  Haltet
euch mit Kritik also nicht zurück, denn jede Anregung für die kommenden
Auf\/lagen ist herzlich willkommen.

Was noch bleibt, ist, euch zur Mitarbeit in der Fachschaft einzuladen.  Die
Fachschaft ist ein "`lockerer Haufen"' Info-Studis, die sich
jede Woche treffen, um die Fachschaftssitzung abzuhalten.
Dort geht es vor allem um das WSI, die Mathematisch-Naturwissenschaftliche
Fakultät und die Universität Tübingen, aber auch um ganz
allgemeine Themen. Kommt doch einfach mal mit euren Fragen und Anregungen
oder einfach mit etwas Neugier auf einem unserer Treffen vorbei.  Zeit
und Ort erfahrt ihr von unserer Webseite oder direkt von uns.

\bigskip

Viel Spaß beim Studieren,

Eure Fachschaft
%\email{fsi@fsi.uni-tuebingen.de}
\vfill
\bigskip

\eject
