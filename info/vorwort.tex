Liebe Erstsemester,

auf den folgenden Seiten haben wir versucht euch das Wichtigste für eine
erste Orientierung im Uni-Alltag zusammenzutragen. Dieses Heft
ist also vor allem eine Sammlung allerlei wichtiger Infos und Tipps\footnote{
Bitte beachtet: Auch wir können uns irren, rechtlich verbindlich ist nur die
jeweilige Studienordnung!}, die ihr euch sonst selbst mühsam durch viele Fragen
und Rennerei zusammensuchen müsstet.

% \coronabox{
%   Dieses Semester soll nach Planung als normales Semester stattfinden. Corona
%   soll keine Rolle mehr spielen. Aber das kann sich ja über Nacht ändern.

%   Aus den letzten Semestern wissen wir, dass es wichtig ist andere Studierende zu
%   finden. Alleine kommt man im Studium nicht weit. Gerade in Mathe sind
%   Übungsgruppen unerlässlich.

% 	Aktuelle Informationen zu Corona-Regelungen findet ihr auf der Webseite\\
%   der Uni\footnote{\url{https://uni-tuebingen.de/universitaet/infos-zum-coronavirus/}}.
% 	Aber es sollte nicht schaden eine stabile Internet-Verbindung, eine Webcam, ein
%   Mikrofon und ein Headset griffbereit zu haben.}

Ihr haltet bereits die \number\auflage. Auflage unseres Info-Heftes in den
Händen. Trotzdem mag das ein oder andere noch zu verbessern sein. Haltet
euch mit Kritik also nicht zurück, denn jede Anregung für die kommenden
Auf\/lagen ist herzlich willkommen.

Was noch bleibt, ist, euch zur Mitarbeit in der Fachschaft einzuladen.  Die
Fachschaft ist ein "`lockerer Haufen"' Info-Studis, die sich einmal wöchentlich
für eine Fachschaftssitzung treffen. Dort geht es vor allem um den Fachbereich,
die Mathematisch-Naturwissenschaftliche Fakultät und die Universität Tübingen, 
aber auch um ganz allgemeine Themen. Kommt doch einfach mal mit euren Fragen 
und Anregungen oder einfach mit etwas Neugier auf einem unserer Treffen vorbei.  
Zeit und Ort erfahrt ihr von unserer Webseite oder direkt von uns.

\bigskip

Viel Spaß beim Studieren,

Eure Fachschaft
%\email{fsi@fsi.uni-tuebingen.de}
\vfill
\bigskip

\eject
