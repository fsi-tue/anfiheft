%\coronabox{Momentan\footnote{Stand: 30.03.2022} sind alle Mensen geöffnet. Dies
%kann sich aber je nach Pandemielage wieder ändern. Ähnliches gilt bei den
%Cafeten.}

In Tübingen gibt es zwei große und eine kleinere Mensa, welche vom
Studierendenwerk betrieben werden. Hier zahlt ihr normalerweise bargeldlos mit
eurem Studierendenausweis. 
Dazu könnt ihr euren Studierendenausweis mit Geld aufladen. Das geht entweder an 
den dafür gedachten Aufwerterstationen, die überall verteilt sind, oder 
per EC-Karte an den Kassen der Mensen und Cafeten. Am einfachsten ist es aber,
\textbf{Autoload}\footnote{\url{https://www.my-stuwe.de/mensa/bezahlung/autoload/}} zu nutzen. 
Dann bucht das StuWe automatisch Betrag X von eurem Bankkonto ab, sobald euer Guthaben unter Betrag Y 
sinkt, wobei X und Y von euch festgelegt werden.

Die Zeiten der Essensausgaben der Mensen sind:
\begin{center}
	\begin{tabular}{ccc}
		Mensa Morgenstelle & Mensa Wilhelmstraße & Mensa Prinz Karl \\
		Mo--Fr, 11:30--14:00 Uhr & Mo--Fr, 11:15--14:00 Uhr & \textit{leider aktuell geschlossen} \\
	\end{tabular}
\end{center}

Die Speisepläne findet ihr auch auf der
Internetseite\footnote{\url{https://www.my-stuwe.de/mensa/}} des
Studierendenwerks oder in der \emph{my-stuwe}-App.

Neben den drei Mensen gibt es noch weitere Cafeterien mit unterschiedlich
großem Angebot. Auf der Morgenstelle gibt es beispielsweise in der Cafeteria
unter der Mensa neben belegten Brötchen sowie Kaffee und Kuchen auch Burger,
Pommes oder Currywurst.\\ Neben dem kulinarischen
Aspekt bieten die Cafeterien häufig auch Arbeitsplätze mit Steckdosen und
WLAN.\\
%\vfill \pagebreak
In Tübingen gibt es folgende Cafeterien:
\begin{itemize}
	\setlength\itemsep{0.3em}
	\item Cafeteria Unibibliothek
	\item Cafeteria Wilhelmstraße
	\item Cafeteria Morgenstelle (im Mensagebäude)
%	\item \textcolor{gray}{Cafeteria Hörsaalzentrum} (Die Flachpfeifen haben die einfach geschlossen!)
%	\item Cafeteria Clubhaus
	\item Cafeteria Theologicum
	\item Cafeteria Neuphilologicum (Brechtbau)
\end{itemize} 
Eine Besonderheit ist die Cafeteria Wilhelmstraße, die
seit dem WiSe 25/26 das \emph{[koeri]werk} beherbergt. Dort bekommt ihr wirklich
leckere Currywurst (nach Wahl vom Rind oder vegan) zu einem guten Preis. Außerdem bietet
die Cafeteria Wilhelmstraße als einzige StuWe-Einrichtung in Tübingen auch Samstags
warme Mahlzeiten an, in der Regel sogar bis 16 Uhr. 
