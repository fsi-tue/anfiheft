% infra/mailinglisten.tex

Die FSI betreibt einige Mailinglisten, die allen Informatikstudierenden die Möglichkeit bieten, Fragen an andere Informatikstudierende zu richten, deren Fragen zu beantworten und sonstige Informationen weiterzuleiten.  Jeder ist dazu eingeladen die Mails, welche an diese Listen gehen zu lesen und selbst an eine Liste zu schreiben. Wir empfehlen euch grade für die erste Zeit eine Anmeldung.

\newcommand{\mladressen}[1]{
    Anmelden: Leere Mail an {\footnotesize \email{#1-subscribe@fsi.uni-tuebingen.de}} \\
    Abmelden: Leere Mail an {\footnotesize \email{#1-unsubscribe@fsi.uni-tuebingen.de}} \\
    Hilfetext: Mail mit Betreff \emph{help} an {\footnotesize \email{#1-request@fsi.uni-tuebingen.de}}}

\begin{description}

  \item[info-studium\At fsi.uni-tuebingen.de (Studium)] ~\\
    Hier geht es um alle Themen, die irgendwie mit dem Informatik-, Bioinformatik- oder Medieninformatik-Studium
    oder dem informatischen Teil der Kognitionswissenschaft zu tun haben. 

    \mladressen{info-studium}

  \item[info-talk\At fsi.uni-tuebingen.de (Laber)] ~\\
    Diese Liste dient der allgemeinen Kommunikation zwischen Informatikern.
    Hier landen u.a. Diskussionen, die auf einer der anderen Listen begonnen
    haben und dort thematisch nicht mehr passen.

    \mladressen{info-talk}

  \item[info-jobs\At fsi.uni-tuebingen.de (Stellenangebote)] ~\\
    Wer auf der Suche nach einem Job ist, sollte sich auf dieser Verteilerliste
    für Stellenangebote anmelden. Achtung, auf dieser Liste sollten nur die
    Angebote selbst und keine Diskussionen darüber landen.

    \mladressen{info-jobs}

	\pagebreak

  \item[kogwiss\At fsi.uni-tuebingen.de (Kognitionswissenschaft)] ~\\
    Auf dieser Liste geht es um Themen, die lediglich Kognitionswissenschaftler
    betreffen.

    \mladressen{kogwiss}
  
  \item[versuche\At fsi.uni-tuebingen.de (Teilnahme an Versuchen)] ~\\
    Wer an wissenschaftlichen Versuchen und Studien teilnehmen will bzw. muss (z.B. Kognitionswissenschaftler,
    die Versuchspersonenstunden ableisten müssen), findet auf dieser 
    im WS15/16 neu eingerichteten Mailingliste Einladungen zur Teilnahme an Versuchen.
  
  \mladressen{versuche}
    
  \item[sport\At fsi.uni-tuebingen.de (Sport)] ~\\
  	Wer ab und zu Lust dazu hat, auf dem Sand Sport zu machen (Volleyball, Fußball, ...) findet auf dieser Mailingliste leicht Menschen die mitspielen wollen, oder erfährt wann man selbst mitspielen kann.
  	
  	\mladressen{sport}

\end{description}

Für die Experten: Mails von diesen Listen filtern:

Die beste Möglichkeit ist, den 'List-Id:'-Header zu verwenden.

% WS2015/16 versuche-mailingliste, typos gefixt - tim
