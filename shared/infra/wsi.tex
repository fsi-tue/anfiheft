% infra/wsi.tex

Die WSI-Studentenpools auf der Morgenstelle und dem Sand sind
  mit euren ZDV-Zugangsdaten zugänglich.

Eine kleine Übersicht über die Dienste der Informatik:

\begin{description}

  \item[Internet:]
    Von den Pool-Rechnern aus habt ihr vollen Zugriff auf's Internet (soweit
    nicht vom ZDV gefiltert, P2P geht nicht).

  \item[Doku, FAQ:]
    Gibt's auf \url{http://helpdesk.informatik.uni-tuebingen.de/}.\\
    Unbedingt lesen, auch wenn ihr sonst keine Handbücher mögt. Eine
    Nichtbeachtung kann euch eine Menge Ärger einbringen! Erweiterungen und
    Ver\-bes\-ser\-ungs\-vor\-schläge sind aber immer willkommen.

  \item[Betriebssystem:]
    Die meisten Pools laufen auf Linux, Windows werdet ihr nur in ZDV-Pools finden. Falls ihr noch nie mit Linux gearbeitet habt, 
    kann ein \textit{Linux Anfängerkurs} sicherlich nicht schaden. Diese werden entweder vom ZDV, den Pool-Admins oder auch von uns angeboten.

  \item[Pools:] ~\\
    \begin{tabular}{llll}
    \multicolumn{3}{l}{Morgenstelle:} \\
    & H33 (C-Bau, 2. Stock) &  SUSE-Linux & 24 Rechner\\
    & H13 (C-Bau, 2. Stock) &  SUSE-Linux & 9 Rechner, Laptoparbeitsplätze\\
    & N03 (C-Bau, 2. Stock) &  SUSE-Linux & 15 Rechner\\
    & P13 (neben N03) (C-Bau, 2. Stock) &  Windows & ~8 Rechner\\

    \multicolumn{3}{l}{Sand:} \\
    &  Sand 13, B033 (UG) & SUSE-Linux & 6 Rechner, Laptoparbeitsplätze\\
    &  Sand 14, C119 (EG) & SUSE-Linux & 2 Rechner (studentischer Arbeitsraum)
    \end{tabular}

  \item[Drucker:]
     In jedem Pool gibt es einen Laserdrucker, der doppelseitig drucken kann.
     Wichtig: Die Drucker sind nicht zum Drucken
     von Skripten gedacht! Sie wären dann für Stunden belegt und eure
     Kommilitonen hätten keinen Toner mehr für die Abgabe ihrer Aufgabenblätter.
     Wer fair ist druckt seine Skripte also zuhause oder im Copyshop, genau wie
     die Studenten aller anderen Fakultäten auch!

     Achtung: PostScript-Drucker, unbedingt Doku beachten (s.o.)!

  \item[Ansprechpartner Pools:]
    Bei Problemen mit den Pools (z.B. auch wenn der Toner im Drucker
    leer ist) könnt ihr euch an \email{helpdesk@informatik.uni-tuebingen.de}
    wenden.  Vorher bitte erst auf \\\url{http://helpdesk.informatik.uni-tuebingen.de/}\\ nachgucken
    (siehe Abschnitt Doku, FAQ).  Je mehr unnötige Fragen (im Sinne von:
    Steht bereits ausführlich in der Doku oder der FAQ) beantwortet werden müssen,
    desto weniger Zeit kann in die Verbesserung des Angebots gesteckt
    werden.  Das heißt jedoch nicht, dass ihr nicht fragen sollt, ganz im
    Gegenteil!  Wenn die Admins nichts von den Problemen wissen, können sie
    diese auch nicht beheben.

  \item[Allgemeine Ansprechpartner]
    Bei Fragen aller Art könnt ihr euch an euren Info I-Tutor oder an die
    Fachschaft wenden.  Wir haben immer ein offenes Ohr für euch. :)


\end{description}
