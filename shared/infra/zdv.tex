% infra/zdv.tex
Jeder Student der Universität Tübingen hat bei seiner Immatrikulation einen ZDV-Account erhalten (meist in der Form zxabc12), der in den Poolräumen der Uni, zur Authentifizierung im WLAN-Netz und für diverse andere Dienste kostenlos benutzt werden kann. 

% \coronabox{Wie bei allem gerade lässt sich nicht sagen, inwieweit die Pool-Räume zur Verfügung stehen.}
\begin{description}

  \item[Drucker and friends:]
    Gedruckt wird auf Laserdruckern der Firma Morgenstern, an denen man mit dem Studierendenausweis bezahlt. Kosten pro S/W-Seite:
    3 Cent.

  \item[E-Mail:]
    Ein Viren-gefilterter (lässt sich nicht abstellen) und SPAM-gefilterter
    (abschaltbar) E-Mail-Account \email{vorname.nachname@student.uni-tuebingen.de},
    der per IMAP sowohl an der Uni als auch Zuhause nutzbar ist. Webmail ist unter \url{https://webmail.uni-tuebingen.de/} verfügbar. 

  \item[Pools:] ~\\
    \begin{tabular}{ll}
      Wilhelmstr. 106 / Wächterstr. 76 &                SUSE-Linux, Windows \\
      Morgenstelle C-Bau, 2. Stock &            SUSE-Linux, Windows \\
    \end{tabular}

    Die Website des ZDV bietet hierzu auch eine Übersichtskarte
    \footnote{\url{https://uni-tuebingen.de/de/1711}}.	%TODO insert \link{}{}?
    Wichtig: Ihr benötigt euren Studierendenausweis, um in der Wilhelmstraße in das Gebäude zu kommen \emph{und} um wieder raus zu kommen! Dafür kommt ihr von 7 Uhr morgens bis 4 Uhr nachts ins Gebäude. Offiziell ist es nicht erlaubt, für jemanden die Tür offen zu halten\footnote{Ansonsten droht die Sperrung des Accounts}, jeder sollte also mit seinem eigenen Ausweis einzeln durch die Türen gehen. Auf der Morgenstelle gelten die Gebäudeöffnungszeiten.

  \item[Kurse:]
    Das ZDV bietet einige Kurse, die z.T. sehr nützlich sind. Z.B. \LaTeX,
    Unix-Ein\-führ\-ung, C-Programmieren, MATLAB ... Nähere Infos erhaltet ihr im
    WWW, per allgemeiner ZDV-Rundmail oder durch Aushänge.

   
   \item[Windows \& Office für Bildung:] Nicht direkt ein Service des ZDV, aber vielleicht trotzdem interessant: Es existiert eine kostenlose Möglichkeit, als Student ein Office 365-Abo und Windows 10 Edu zu erwerben. Nachteil: Man muss ein separates Konto bei Microsoft anlegen. Weitere Infos hierzu findet ihr unter \url{https://uni-tuebingen.de/de/221199}.	%TODO insert \link{}{}?


  \item[Bildungslizenzen:] Auf der Software Übersichtsseite \footnote{\url{https://uni-tuebingen.de/de/77076}} kann man sich durch die verschieden Software Lizenzen klicken, die auch Studierenden kostenlos zur Verfügung stehen.

  \item[Ansprechpartner Pools:]
    Ist erreichbar unter \email{beratung@zdv.uni-tuebingen.de}.
    %\footnote{siehe auch \url{http://www.zdv.uni-tuebingen.de/kontakt-antraege-beratung.html}}.

	
\end{description}
