Sollte einem Tübingen irgendwann mal langweilig geworden sein, kann man auch
mal ein Semester im Ausland verbringen. Die Kognitionswissenschaft unterhält
dazu zusammen mit der Informatik Erasmus-Austauschprogramme mit mehreren
europäischen Universitäten. Informationen zum Bewerbungsverlauf und den
verschiedenen Partneruniversitäten findet ihr unter
\url{https://uni-tuebingen.de/de/113267}.\\	%TODO insert \link{}{}?

Als geeignete Semester bieten sich hierfür das fünfte und sechste Semester an,
da man hier fast ausschließlich Wahlpflichtveranstaltungen hat, für die sich
auch leicht Leistungen aus dem Ausland anerkennen lassen. Sollte man bereits in
einem früheren Semester ins Ausland wollen, sollte man mit dem
Prüfungsausschuss abklären, welche Veranstaltungen man sich für welche
Pflichtveranstaltung anrechnen lassen kann.

Generell empfiehlt es sich, sich möglichst früh um einen Auslandsplatz zu
kümmern. Je nachdem wo man hin möchte sind die Plätze schon über ein Jahr im
Voraus belegt. Dies gilt vor allem, wenn man einen Austausch mit einer
Universität im nicht europäischen Ausland machen möchte.
