% nebenfaecher/bwl.tex

Früher konnte man BWL oder VWL als Nebenfach studieren. Mittlerweile ist das für Bachelor anders, es gibt keine Unterscheidung mehr. Ihr könnt nur Wirtschaftswissenschaften als Nebenfach machen, erst im Master gibt es BWL bzw. VWL. Es gibt auch keine Vorlesungen zur Auswahl, entscheidet ihr euch für WiWi, müsst ihr "`Einführung in die Wirtschaftswissenschaft"' und "`Grundlagen der Volkswirtschaftslehre"' hören. 

Die Vorlesungen finden üblicherweise im Kupferbau oder in der Neuen Aula statt. Ihr solltet den Tü-Bus also gleich in euren Stundenplan mit einbauen, denn freie Parkplätze unten in der Stadt werdet ihr zu Vorlesungszeiten vergeblich suchen. Wer einen guten Sitzplatz im Hörsaal ergattern möchte, muss sich zudem zeitig einfinden, denn selbst die 600 Plätze des HS25 im Kupferbau sind irgendwann alle belegt.
Das Prüfungsamt findet ihr in der Haußerstr. 11 und hat folgende Öffnungszeiten:	Mo und Mi 14--16 Uhr, 	Di und Do 9--11 Uhr. Wollt ihr euch aber einfach für die Prüfungen anmelden, könnt ihr auch das Formular in den Briefkasten werfen. Da sind die WiWis irgendwie entspannter als wir.

Informationen zu allem, was die WiWis betrifft, gibt es im Internet unter der offiziellen Homepage: \url{http://www.wiwi.uni-tuebingen.de/}.  

Lohnend ist auch ein Besuch bei der Fachschaft WiWi in der Mohlstraße 36. Ihr findet dort Skripte zu den Vorlesungen, alte Klausuren, zumeist mit Musterlösungen und vieles andere, was einem das Studileben sonst noch so erleichtern kann.  Die Sprechstunden der Fachschaft und einen Teil der Skripte und Klausuren gibt es auch auf den Fachschaftsseiten im Netz: \url{http://www.ffw.uni-tuebingen.de/}.

Während des Semesters organisiert die Fachschaft WiWi regelmäßig eine Bücherflohmarkt. Dort kann man sich mehr oder weniger günstig mit meist älteren Ausgaben studienrelevanter Bücher eindecken. Ein Besuch lohnt auf jeden Fall!
