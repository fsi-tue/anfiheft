%% aktualisiert: 10.10.2009

In diesem Modul kriegt ihr die Grundlagen eines Teilgebiets der Rechtwissenschaft vermittelt mit den Teilgebieten Öffentliches Recht, Strafrecht oder Zivilrecht. Dabei müsst ihr eines dieser Teilgebiete auswählen in denen ihr dann die jeweiligen Veranstaltungen belegt. So erwarten euch im "`Öffentliches Recht"' die Veranstaltungen "`Öffentliches Recht I: Staatsorganisation mit Fallbesprechungen"', "`Europarecht"' und "`Öffentliches Recht II: Grundrechte"'. Im Zivilrecht habt ihr dann "`Zivilrecht I mit Fallbesprechungen"', "`Zivilrecht II: Schwerpunkt Schuldrecht mit Fallsprechungen"' und "`Übungen im Zivilrecht für Anfänger"'. Im 3. Teilgebiet dem Strafrecht habt ihr "`Strafrecht I: Allgemeiner Teil"', "`Strafrecht II: Besonderer Teil 1"' und "`Strafrecht III: Besonderer Teil 2"' zu belegen.

Beachtet bitte, dass das Studium einer Teildisziplin der Rechtswissenschaften Zulassungsbeschränkt ist und ihr euch dafür anmelden müsst.

Weitere Informationen gibt es beim Studienfach\-be\-rater (\email{nebenfach@jura.uni-tuebingen.de}) oder auf der Website\footnote{\url{http://www.jura.uni-tuebingen.de/studium/nebenfach}}.

%Ansprechparter aus der Informatik ist Prof. Schweizer\footnote{\email{Harald.Schweizer@uni-tuebingen.de}}.
