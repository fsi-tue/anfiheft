% studium/studium.tex

Als Studenten in einem der Informatikstudiengänge besucht ihr im
  Bachelorstudium Vorlesungen, Übungen, Proseminare, Praktika
  und selten sogar Exkursionen.

In den \textbf{Vorlesungen} wird der Lehrstoff in mehr oder weniger
  ausgearbeiteter Form vom Dozenten vorgetragen.  Meistens gibt
  es ein Skript oder begleitendes Buch, so dass ihr nicht alles mitschreiben 
  müsst.  Die Skripte finden sich meistens im Netz, entsprechende Adressen 
  werden in der Veranstaltung bekannt gegeben. Oft findet sich auch ein 
  regelmäßiger Mitschreiber, der seine Mitschriebe allgemein zur Verfügung 
  stellt. Den erlernten Stoff solltet ihr auf jeden Fall zu Hause nochmals 
  nacharbeiten, denn das Tempo, das die Dozenten vorlegen, ist meist recht 
  schnell. Trotzdem ist es meistens beim Lernen auf Prüfungen sehr hilfreich, 
  die Vorlesungen besucht zu haben - sie bieten einen Überblick über das 
  Themengebiet, helfen es zu strukturieren und man merkt, welche Schwerpunkte 
  der Professor setzt.

\textbf{Übungen} sind Unterrichtsveranstaltungen mit begrenzter
  Teilnehmerzahl, die den Stoff einer bestimmten Vorlesung vertiefen
  sollen. Sie finden begleitend und oft verpflichtend zu den
  Vorlesungen statt. Die Tutoren -- üblicherweise Studenten aus höheren
  Semestern -- sprechen mit euch vor allem die Korrektur eurer
  Lösungen zu den wöchentlichen Übungszetteln durch.  Außerdem
  werden Tipps für die Aufgaben der nächsten Woche gegeben und
  allgemeine Fragen zum Stoff beantwortet.  Wichtig ist, dass \textbf{ihr}
  etwas aus der Übung macht -- ihr werdet bald merken, dass eine gute
  Übung zum Verständnis des Stoffes weit mehr beiträgt als die
  zugehörige Vorlesung.  Allerdings dürft ihr keine Scheu haben,
  nachzuhaken, wenn euch etwas unklar ist.  Die Übungsgruppenleiter
  sind auch "`nur"' Studenten und können eure Probleme daher gut
  verstehen.  Außerdem werden sie für eure Fragen bezahlt!

In \textbf{(Pro-)Seminaren} halten die einzelnen Teilnehmer jeweils ein
  Referat zu einem bestimmten Thema und erstellen im Anschluss eine schriftliche
  Ausarbeitung.  Im Bachelor müsst ihr ein Proseminar belegen.  Im
  Vergleich zu dem Seminar, welches ihr für den Master belegen müsst, sollten 
  die Anforderungen im Proseminar etwas niedriger sein. Es empfiehlt sich ab
  dem zweiten Semester die Augen nach einem interessanten Thema offen zu
  halten, da nicht alle Themen jedes Jahr angeboten werden.

\enlargethispage{10pt}
Im Informatikstudium bezeichnet ein \textbf{Praktikum} nicht eine
  berufsbezogene praktische Tätigkeit außerhalb der Universität,
  sondern eine Veranstaltung, in der wissenschaftliche Experimente zu
  Übungszwecken durchgeführt werden.  Beispielsweise muss in einigen der 
  Studiengänge das Basispraktikum zur Technischen Informatik oder der 
  Tierphysiologische Kurs besucht werden.  Hier wird erworbenes theoretisches 
  Wissen dadurch vertieft, dass es aktiv angewandt wird. Wenn man sich auf die 
  Praktika richtig vorbereitet, sind die einleitenden Abfragen zum einen viel 
  leichter, aber vor allem lernt man wesentlich mehr, wenn man weiß, welchen 
  Effekt man gerade untersucht.

\textbf{Exkursionen} sind Studienfahrten oder andere Veranstaltungen
  außerhalb der Universität, die den Lehrstoff durch Besichtigung,
  praktische Untersuchungen und unmittelbare Anschauungen ergänzen.
  Sie sind aber naturgemäß eher selten in der Informatik. Mitunter ist es
  der Fachschaft aber auch schon mal gelungen, den einen oder anderen Dozenten 
  zu einer Exkursion zu überreden (zum Beispiel in das Deutsche Museum 
  München).

\bigskip


Mit all diesen Veranstaltungen könnt ihr die in eurem Modulhandbuch angegebenen 
\textbf{Module} füllen - oftmals gibt es da recht genaue Vorgaben (wider 
Erwarten kann man nicht sein ganzes Studium nur mit Exkursionen füllen - dafür 
hätte man zum Beispiel Biologie studieren müssen). Ihr findet eure 
Prüfungsordnung und die Modulhandbücher auf der Webseite der Fakultät.

Inzwischen laufen alle \textbf{Prüfungen} studienbegleitend ab. Das heißt, dass 
es nicht eine große Bachelorprüfung nach drei Jahren gibt, sondern jede 
Veranstaltung in unmittelbarer zeitlicher Nähe geprüft wird. Die Termine geben 
die Dozenten in den Veranstaltungen bekannt - fragt unbedingt nach, wenn das 
nicht in den ersten zwei Wochen passiert sein sollte (dazu sind sie nämlich 
verpflichtet)!

Um dann aber an der Prüfung im Anschluss an die Veranstaltung teilnehmen zu 
dürfen, müsst ihr üblicherweise eine ganze Reihe von \textbf{Auflagen} 
erfüllen, die die Dozenten recht frei festlegen können. Dazu zählt eigentlich 
immer, dass man eine festgelegte Anzahl an Übungsaufgaben korrekt bearbeiten 
muss. Dazu muss man üblicherweise ein paar Mal in der Übungsgruppe vorrechnen oder 
dem Tutor seine Lösung im Einzelgespräch genau erklären.

%\pagebreak
Wo wir gerade bei diesem unangenehmen Thema sind: Zu den Prüfungen
  müsst ihr euch natürlich anmelden. Das ist für einige Vorlesungen online im 
  CAMPUS möglich - für alle anderen müsst ihr die Anmeldung jedoch in jedem 
  Fall auch schriftlich im Prüfungssekretariat abgeben. Zudem k\"onnen je
  nach Fach (insbesondere bei Nebenf\"achern) noch weitere Schritte hinzukommen.
  Das Prüfungssekretariat für Bioinformatiker und Medizininformatiker befindet sich auf dem Sand bei
  Frau Weber (Sand 13,  B316), das f\"ur 
  Informatiker und Medieninformatiker bei Frau Hallmayer (Sand 13, B118) und das f\"ur
  Kognitionswissenschaftler bei Frau Hutt (Morgenstelle E-Bau, 3A34).

Als Hilfe für die Vorbereitung gibt es bei uns ein Onlinesystem mit Protokollen\footnote{\url{https://ppi.fsi.uni-tuebingen.de/}} 
  über bisherige Prüfungen. Redet 
  früh genug bei Problemen mit dem Prof oder mit einem der Studienberater. Sie 
  helfen euch auch sonst immer gerne.

Das größte Hindernis sind für Anfänger in der Regel die
  Mathe-Vorlesungen und die damit verbundenen Übungsaufgaben.  In sie
  werdet ihr im ersten Jahr mit die meiste Zeit investieren. Die meisten
  werden schnell merken, dass es oft schwierig ist, die Aufgaben
  allein zu lösen. Doch dadurch dürft ihr euch nicht entmutigen lassen, da 
  diese Übungsblätter nicht zum alleine lösen konzipiert wurden. Die 
  Klausuraufgaben sind in der Regel einfacher, weil sie sich eher auf Anwendung 
  eures Wissens beschränken.

Für alle Übungen gilt, dass ihr von Anfang an kleine Gruppen bilden solltet,
  um die Aufgaben gemeinsam zu lösen.  In der Gruppe kann man zusammentragen,
  was die Einzelnen sich überlegt oder in Büchern gefunden haben,
  und es besteht die Möglichkeit, sich etwas erklären zu lassen
  oder über Lösungsmöglichkeiten zu diskutieren.



Selbst diejenigen, die erklären, lernen dabei viel und werden im Stoff sicherer. 
  Auch das Übernehmen von Lösungen in eigenen Worten ist legitim, solange
  ihr den Lösungsweg wirklich versteht, was der Tutor durch Vorrechnen abfragen 
  kann. Schaut euch an wie die Aufgabe gelöst wurde und schreibt es dann blind 
  in eigenen Worten neu auf.

Die Zeit, die ihr außerhalb der Veranstaltungen für Übungen aufwenden
  müsst, ist es durchaus wert - habt ihr die Übungen selbst gemacht und 
  nachvollzogen, fallen euch die Prüfungen viel leichter.

\pagebreak
  Ein Hinweis noch: Ihr seid zwar nicht mehr die ersten, die auf den Abschluss 
  Bachelor studieren, manchmal wird es aber trotzdem nicht ausbleiben, 
  dass es Probleme gibt (besonders für die Kognitionswissenschaftler und 
  Medizininformatiker). Das liegt vor allem daran, dass auch die Profs laufend 
  versuchen, gefundene Probleme zu lösen - 
  dafür dann aber mitunter an anderer Stelle neue auftreten, an die 
  bisher noch niemand gedacht hat. In diesem Fall ist ein bisschen 
  Eigeninitiative wichtig:
  Es ist sehr gut möglich, dass bestimmte Dinge einfach untergehen, wenn 
  man sich nicht selbst darum kümmert - ihr erlebt den Studienalltag 
  schließlich hautnah. 
  Wendet euch in diesem Fall an eure Kommilitonen, die Fachschaft, die 
  Studienberater oder den Studiendekan Prof. \Studiendekan.
  Irgendwo kann euch sicherlich geholfen werden und vielleicht sind hinterher 
  eure Kommilitonen froh, dass es ein Problem weniger gibt! Grundsätzlich gilt: 
  Wenn ihr niemandem von dem gefundenen Problem erzählt, wird sich auch nichts 
  ändern!


%%% Local Variables:
%%% mode: latex
%%% TeX-master: "ethik"
%%% End:
