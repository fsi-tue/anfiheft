\subsubsection*{Studierendensekretariat}
Das Studierendensekretariat\footnote{\url{https://uni-tuebingen.de/de/596}} befindet sich in der Wilhelmstraße 11 direkt neben der ehemaligen Mensa Wilhelmstraße. Das Studierendensekretariat ist zuständig für eure Immatrikulation und Exmatrikulation, amtliche Bescheinigungen, sowie alles was mit dem Studierendenausweis zu tun hat.	%TODO insert \link{}{}?

\subsubsection*{Studierendenterminals}
An den Studierendenterminals könnt ihr den Aufdruck auf eurem
Studierendenausweis aktualisieren. Dies solltet ihr nach erfolgreicher Rückmeldung schleunigst tun, oft wird beim Kauf des Semestertickets ein aktueller Aufdruck erwartet, außerdem müsst ihr sonst in den Mensen
den vollen Gästepreis zahlen!\\
Momentan gibt es drei Terminalgruppen:\\
Eine auf der Morgenstelle im Hörsaalzentrum in der Nähe der Damentoilette,
eine im Studentensekretariat in der Wilhelmstraße und eine bei den Kopierern in der UB.

\subsubsection*{CIP Pools}
Räume mit für Studierende zugänglichen Computern befinden sich sowohl auf der Morgenstelle (Untergeschoss des C-Bau) als auch auf dem Sand (in den Räumen B033 und C119). Auf den Rechnern in diesen Räumen könnt ihr euch mit eurer ZDV-Kennung und Passwort unkompliziert anmelden.

\subsubsection*{Bibliotheken}
\textbf{DIE} Bibliothek der Universität ist die
Universitätsbibliothek (UB).  Im Informationsheft der ZV sind
sämtliche ihrer Einrichtungen auf\-ge\-führt.  Sie hat
allerdings den Nachteil, dass sie die einzelnen Fachbereiche nicht so
detailliert abdecken kann, wie es manchmal wünschenswert wäre,
denn dafür sind die Fach-Bibliotheken der einzelnen Fakultäten
zuständig.  Aber mit etwas Aufwand kann über die UB fast jedes Buch
(auch per Fernausleihe) besorgt werden -- Fragen schadet nie!
%\pagebreak
\begin{center}
\includegraphics[width=0.45\hsize]{shared/xkcd/books_toomany.png}
\end{center}

%\pagebreak

Für die Kognitionswissenschaft sind folgende Bibliotheken interessant:
\begin{description}
	\item[Hauptstelle der Unibibliothek] Zwar sind hier nicht eure wichtigsten
	Bücher untergebracht, aber im neuen Ammer-Bau lässt es sich prima
	lernen. Es gibt zahlreiche Arbeitsplätze und Arbeitsräume, die
	allerdings schnell belegt sind. Außerdem findet man hier immer die
	aktuellsten Tageszeitungen und einige Rechnerpools.
	Grade in den ersten Tagen eures Studiums ist ein Besuch der Hauptstelle
	sehr zu empfehlen, da zu Beginn des Semesters täglich Einführungen in den Umgang mit der UB stattfinden. Zusätzlich gibt es viele hilfreiche Tricks für das tägliche Leben mit der UB. Wann diese Führungen angeboten werden, erfahrt ihr unter  \url{http://www.ub.uni-tuebingen.de/}. Wichtig sind die extrem langen Ausleihfristen: die Standardfrist von 4 Wochen verlängert sich automatisch auf bis zu 3 Monate, falls euer Buch nicht angefordert wird. Dafür solltet ihr den EMail - Alarm anschalten, damit ihr auch bei Abwesenheit aus Tübingen über Rückgabefristen informiert bleibt. 
	
	\item[Außenstelle der Uni-Bibliothek auf der Morgenstelle]
	Hier findet ihr eigentlich alles, was ihr im Grundstudium an
	Literatur braucht.  Was nicht in den Regalen steht, könnt ihr
	im Normalfall bestellen.  Die meisten der dort stehenden Bücher könnt ihr
	ausleihen. Den dazu benötigten Ausweis besitzt ihr in Form eures Studierendenausweises bereits.
	
	\item[Mathe-Bibliothek auf der Morgenstelle]
	Das ist die Fachbibliothek der Fakultät für Mathematik. Sie
	befindet sich im C-Gebäude auf dem 4. Stock (vom
	Hörsaalzentrum aus kommt man schon im 3. Stock in das
	C-Gebäude!). Einige Bücher können ausgeliehen werden,
	andere sind nur Präsenzbestand.  Außerdem steht da ein Regal
	mit Informatik-Büchern und Skripten, die z.T. ganz
	interessant sind (gleich links vom Eingang).  Auch finden sich
	hier die sog. Hand-Apparate zu den Vorlesungen (nach dem
	Eingang rechts und dann ganz nach hinten).  Dies sind
	Zusammenstellungen von Büchern, die ein Prof. zu seiner
	Vorlesung empfiehlt.  Diese Bibliothek ist besonders
	interessant, weil hier in Ruhe gearbeitet werden kann.
	Außerdem gibt es einen Kopierer in dieser Bibliothek (linke
	Seite) und zwei kleine Arbeitsräume mit Tafeln aber ohne
	Tageslicht (rechte Seite).  Diese sind für Gruppenarbeiten
	genauso geeignet, wie die Tische und Bänke, die vor der
	Mathe-Bib aufgestellt sind.
	
	\item[Informatik-Bibliothek auf dem Sand]
	Die Fachbibliothek der Informatik befindet sich auf dem Sand
	im ersten Stockwerk.  Die Bücher dort sind eigentlich erst in
	späteren Semestern interessant.  Allerdings steht sie für jeden
	offen, der sich jetzt schon dafür interessiert.  "`Steht
	offen"' ist allerdings mit Vorsicht zu genießen, da die
	Bibliothek nicht immer geöffnet ist. Die aktuellen Zeiten
	sind an der Bib angeschlagen, momentan ist die Bib auf dem Sand lediglich von 14 bis 17 Uhr geöffnet.
	In den Semesterferien ist mit unregelmäßigen, reduzierten Besuchszeiten zu rechnen, manchmal wird die Bib über die Semesterferien auch komplett geschlossen!
	
	Ebenfalls interessant ist die umfangreiche Zeitschriftensammlung,
	die eigentlich alle wichtigen Zeitschriften im
	Informatik-Bereich umfasst.
	
	Für euch sind zunächst insbesondere die Klausuren der
	vergangenen Jahre wichtig, welche im zweiten Raum links (der mit
	der großen Metallschiebetür) auf der linken Seite in Ordnern gesammelt
	werden.  Nach Absprache mit der Aufsicht solltet ihr euch vor den
	Prüfungen einige Klausuren zur Übung kopieren.
	
	Sämtliche vorhandene Medien sind Präsenzbestand. Bücher können
	jedoch über Nacht oder über das Wochenende ausgeliehen werden, wenn über das Terminal ausgeliehen wird.
	
	Weitere Informationen findet ihr unter \url{http://bib.informatik.uni-tuebingen.de}	%TODO insert \link{}{}?
	
	\item[Psychologie-Bibliothek im PI]
Im Untergeschoss des Psychologischen Instituts befindet sich die Psychologie-Bibliothek. Dort findet ihr viele Kogni-relevante Bücher und Kopiervorlagen für die Bücher der Psychologie-Vorlesungen. Außerdem erhaltet ihr dort eure Versuchspersonen-Bescheinigungen.
	
        \item[Fachschaftsbibliothek]
        In der Fachschaftsbibliothek findet ihr für unser Fach interessante Allgemeinliteratur. Auf unserer Website findet ihr eine Bestandsliste\footnote{https://www.fs-kogni.uni-tuebingen.de/bibliothek/}.
\end{description}

Wenn ihr nicht gerade eine Gruppenarbeit erledigen müsst, so sind die
Bibliotheken eigentlich sehr gute Plätze zum Arbeiten.  So hat man
auch die Bücher direkt vor Ort und es besteht kein Bedarf sie
auszuleihen.  Deshalb ist zum Beispiel die Info-Bib auch eine
Präsenzbibliothek, was bedeutet, dass man normalerweise als
Studierender Bücher nicht für länger ausleihen darf.


