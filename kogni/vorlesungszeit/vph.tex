Die Kognitionswissenschaft und die Psychologie sind empirische Wissenschaften, daher finden hier an der Uni viele Experimente statt. Darin wird unser Verhalten unter Laborbedingungen untersucht.\\
Solche Versuche sind häufig auch Teil von Abschlussarbeiten oder Promotionen. Da meistens nicht genügend Geld vorhanden ist, um alle Versuchspersonen ausreichend zu bezahlen, hat es sich in der Kognitionswissenschaft genauso wie in der Psychologie etabliert, Studierende zur Teilnahme an solchen Experimenten zu verpflichten.\\
Ihr müsst daher bis zum Ende des dritten Semesters insgesamt 30 Stunden lang an Versuchen teilgenommen haben. Über die jeweiligen Teilnahmen erhaltet ihr jeweils eine Bescheinigung, die ihr entweder sofort bekommt oder bei der Bibliothekarin der Bibliothek im Psychologischen Institut abholen könnt. Diese Scheine müsst ihr sammeln und Ende des dritten Semesters in der Veranstaltung Experimentelle Kognitionswissenschaft abgeben. Diese werden euch dann als ein Leistungspunkt der Veranstaltung angerechnet.\\
In besagter Veranstaltung werdet ihr auch erstmals selbst ein Experiment durchführen. Es ist deshalb sehr ratsam, schon einen großen Teil der Versuchspersonenstunden davor gemacht zu haben, damit ihr euch schon mit den Abläufen eines Experiments auskennt.